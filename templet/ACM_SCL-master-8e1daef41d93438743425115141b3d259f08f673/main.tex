\documentclass[titlepage,landscape,a4paper,10pt]{article}
\usepackage{listings, color, fontspec, minted, setspace, titlesec, fancyhdr, dingbat, mdframed, multicol}
\usepackage{graphicx, amssymb, amsmath, textcomp}
\usepackage[left=1.5cm, right=0.7cm, top=1.7cm, bottom=0.0cm]{geometry}

\usepackage{xeCJK}
\setCJKmainfont{STKaiti}
\setmainfont{PTMono-Regular}
%\setmainfont{Consolas}

%configure the top corners
\pagestyle{fancy}
\setlength{\headsep}{0.1cm}
\rhead{Page \thepage}
\lhead{DHU}

%configure space between the two columns
\setlength{\columnsep}{30pt}

%configure minted to display codes 
\definecolor{Gray}{rgb}{0.9,0.9,0.9}

%remove leading numbers in table of contents
\setcounter{secnumdepth}{0}

%configure section style
%\titleformat{\section}
%	{\normalfont\normalsize}	% The style of the section title
%	{}					% a prefix
%	{0pt}				% How much space exists between the prefix and the title
%	{\quad}				% How the section is represented
\titleformat{\section}{\large}{}{0pt}{}
\titlespacing{\section}{0pt}{0pt}{0pt}

%enable section to start new page automatically
%\let\stdsection\section
%\renewcommand\section{\penalty-100\vfilneg\stdsection}

%\renewcommand\theFancyVerbLine{\arabic{FancyVerbLine}}
\renewcommand{\theFancyVerbLine}{\small \oldstylenums{\arabic{FancyVerbLine}}}

\setminted[cpp]{
	style=xcode,
	mathescape,
	linenos,
	autogobble,
	baselinestretch=1.0,
	tabsize=4,
	%bgcolor=Gray,
	frame=single,
	framesep=1mm,
	framerule=0.3pt,
	numbersep=1mm,
	breaklines=true,
	breaksymbolsepleft=2pt,
	%breaksymbolleft=\raisebox{0.8ex}{ \small\reflectbox{\carriagereturn}}, %not moe!
	%breaksymbolright=\small\carriagereturn,
	breakbytoken=false,
}

\setminted[java]{
	style=xcode,
	mathescape,
	linenos,
	autogobble,
	baselinestretch=1.0,
	tabsize=4,
	%bgcolor=Gray,
	frame=single,
	framesep=1mm,
	framerule=0.3pt,
	numbersep=1mm,
	breaklines=true,
	breaksymbolsepleft=2pt,
	%breaksymbolleft=\raisebox{0.8ex}{ \small\reflectbox{\carriagereturn}}, %not moe!
	%breaksymbolright=\small\carriagereturn,
	breakbytoken=false,
}
\setminted[text]{
	style=xcode,
	mathescape,
	linenos,
	autogobble,
	baselinestretch=1.0,
	tabsize=4,
	%bgcolor=Gray,
	frame=single,
	framesep=1mm,
	framerule=0.3pt,
	numbersep=1mm,
	breaklines=true,
	breaksymbolsepleft=2pt,
	%breaksymbolleft=\raisebox{0.8ex}{ \small\reflectbox{\carriagereturn}}, %not moe!
	%breaksymbolright=\small\carriagereturn,
	breakbytoken=false,
}

%configure titles
\title{\Large{DHU}}
\date{\today}

%THE SCL BEGINS
\begin{document}
\maketitle

\begin{multicols*}{2}

\begin{spacing}{0}
	\tableofcontents
\end{spacing}
\end{multicols*}

\begin{multicols}{2}

\newpage


\begin{spacing}{0.8}
%杂与数论
\section{汇编快速乘}
\inputminted{cpp}{/Users/Corn/Desktop/Code/ACM_SCL/code/mul_mod.cpp}

\section{快速幂}
\inputminted{cpp}{/Users/Corn/Desktop/Code/ACM_SCL/code/pow_mod.cpp}

\section{扩展欧几里得}
\inputminted{cpp}{/Users/Corn/Desktop/Code/ACM_SCL/code/ext_gcd.cpp}

\section{扩展中国剩余定理}
\inputminted{cpp}{/Users/Corn/Desktop/Code/ACM_SCL/code/ext_crt.cpp}

\section{乘法逆元}
\inputminted{cpp}{/Users/Corn/Desktop/Code/ACM_SCL/code/Inv.cpp}

\section{mint}
\inputminted{cpp}{/Users/Corn/Desktop/Code/ACM_SCL/code/ModInt.cpp}

\section{组合数取模}
\inputminted{cpp}{/Users/Corn/Desktop/Code/ACM_SCL/code/nCm.cpp}

\section{扩展卢卡斯}
\inputminted{cpp}{/Users/Corn/Desktop/Code/ACM_SCL/code/ext_lucas.cpp}

\section{高斯消元}
\inputminted{cpp}{/Users/Corn/Desktop/Code/ACM_SCL/code/Gauss.cpp}

\section{高斯消元-异或}
\inputminted{cpp}{/Users/Corn/Desktop/Code/ACM_SCL/code/Gauss_xor.cpp}

\section{欧拉函数}
\inputminted{cpp}{/Users/Corn/Desktop/Code/ACM_SCL/code/eular_phi.cpp}

\section{素数表}
\inputminted{cpp}{/Users/Corn/Desktop/Code/ACM_SCL/code/primetable.cpp}

\section{素数判定-质因数分解}
\inputminted{cpp}{/Users/Corn/Desktop/Code/ACM_SCL/code/MillerRabin_PollardRho.cpp}

\section{素数个数统计}
\inputminted{cpp}{/Users/Corn/Desktop/Code/ACM_SCL/code/prime_cnt.cpp}

%数据结构
\section{并查集}
\inputminted{cpp}{/Users/Corn/Desktop/Code/ACM_SCL/code/uni_find.cpp}

\section{树状数组}
\inputminted{cpp}{/Users/Corn/Desktop/Code/ACM_SCL/code/BIT.cpp}

\section{线段树}
\inputminted{cpp}{/Users/Corn/Desktop/Code/ACM_SCL/code/segtree.cpp}

\section{树链剖分}
\inputminted{cpp}{/Users/Corn/Desktop/Code/ACM_SCL/code/decomposition.cpp}

\section{ST表}
\inputminted{cpp}{/Users/Corn/Desktop/Code/ACM_SCL/code/ST.cpp}

\section{字符串哈希}
\inputminted{cpp}{/Users/Corn/Desktop/Code/ACM_SCL/code/Rabin_Karp.cpp}

\section{后缀数组}
\inputminted{cpp}{/Users/Corn/Desktop/Code/ACM_SCL/code/suffix_array.cpp}

\section{AC自动机}
\inputminted{cpp}{/Users/Corn/Desktop/Code/ACM_SCL/code/AC_automaton.cpp}

\section{Treap}
\inputminted{cpp}{/Users/Corn/Desktop/Code/ACM_SCL/code/treap.cpp}

\section{动态树}
\inputminted{cpp}{/Users/Corn/Desktop/Code/ACM_SCL/code/LCT.cpp}


%图论
\section{前向星}
\inputminted{cpp}{/Users/Corn/Desktop/Code/ACM_SCL/code/Gra_base.cpp}

\section{最短路-dijk}
\inputminted{cpp}{/Users/Corn/Desktop/Code/ACM_SCL/code/dijk.cpp}

\section{最短路-spfa}
\inputminted{cpp}{/Users/Corn/Desktop/Code/ACM_SCL/code/spfa.cpp}

\inputminted{text}{/Users/Corn/Desktop/Code/ACM_SCL/extra/flow.txt}

\section{最大流-isap}
\inputminted{cpp}{/Users/Corn/Desktop/Code/ACM_SCL/code/maxflow_isap.cpp}

\section{最大流-dinic}
\inputminted{cpp}{/Users/Corn/Desktop/Code/ACM_SCL/code/maxflow_dinic.cpp}

\section{最小费用最大流-dijk}
\inputminted{cpp}{/Users/Corn/Desktop/Code/ACM_SCL/code/mcmf_dijk.cpp}

\section{最小费用最大流-spfa}
\inputminted{cpp}{/Users/Corn/Desktop/Code/ACM_SCL/code/mcmf_spfa.cpp}

\section{最小生成树-ElogV}
\inputminted{cpp}{/Users/Corn/Desktop/Code/ACM_SCL/code/kruskal.cpp}

\section{最小生成树-Prim-V*V}
\inputminted{cpp}{/Users/Corn/Desktop/Code/ACM_SCL/code/prim.cpp}

\section{全局最小割}
\inputminted{cpp}{/Users/Corn/Desktop/Code/ACM_SCL/code/global-min_cut.cpp}

\section{割点与桥、点双联通分量}
\inputminted{cpp}{/Users/Corn/Desktop/Code/ACM_SCL/code/cut_bridge_bcc.cpp}

\section{强联通分量}
\inputminted{cpp}{/Users/Corn/Desktop/Code/ACM_SCL/code/scc.cpp}

\section{LCA-Tarjan}
\inputminted{cpp}{/Users/Corn/Desktop/Code/ACM_SCL/code/LCA_Tarjan.cpp}

\section{LCA-ST}
\inputminted{cpp}{/Users/Corn/Desktop/Code/ACM_SCL/code/LCA_ST.cpp}

%计算几何
\section{平面几何-点}
\inputminted{cpp}{/Users/Corn/Desktop/Code/ACM_SCL/code/Point.cpp}

\section{平面几何-圆}
\inputminted{cpp}{/Users/Corn/Desktop/Code/ACM_SCL/code/circle.cpp}

\section{平面几何-多边形与凸包}
\inputminted{cpp}{/Users/Corn/Desktop/Code/ACM_SCL/code/convex.cpp}

\section{平面几何-半平面交}
\inputminted{cpp}{/Users/Corn/Desktop/Code/ACM_SCL/code/Phalf.cpp}

\section{立体几何-点、线、面}
\inputminted{cpp}{/Users/Corn/Desktop/Code/ACM_SCL/code/Geo3D_Base.cpp}

\section{simpson积分}
\inputminted{cpp}{/Users/Corn/Desktop/Code/ACM_SCL/code/simpson.cpp}

\section{romberg积分}
\inputminted{cpp}{/Users/Corn/Desktop/Code/ACM_SCL/code/romberg.cpp}

%others
\section{Bitset}
\inputminted{cpp}{/Users/Corn/Desktop/Code/ACM_SCL/code/Bitset.cpp}

\section{线段树补充}
\inputminted{cpp}{/Users/Corn/Desktop/Code/ACM_SCL/code/segtree_extra.cpp}

\section{另一个半平面交}
\inputminted{cpp}{/Users/Corn/Desktop/Code/ACM_SCL/code/Phalf_extra.cpp}

\end{spacing}
\end{multicols}


\begin{multicols}{2}
\begin{spacing}{0.8}
\section{geany}
\inputminted{text}{/Users/Corn/Desktop/Code/ACM_SCL/extra/geany.txt}

\section{Java}
\inputminted{cpp}{/Users/Corn/Desktop/Code/ACM_SCL/extra/Main2.java}
\inputminted{cpp}{/Users/Corn/Desktop/Code/ACM_SCL/extra/BigInteger.java}
\inputminted{cpp}{/Users/Corn/Desktop/Code/ACM_SCL/extra/Main.java}

\section{STL}
\inputminted{cpp}{/Users/Corn/Desktop/Code/ACM_SCL/extra/STL.cpp}

\inputminted{cpp}{/Users/Corn/Desktop/Code/ACM_SCL/extra/pb_ds_tree.cpp}
\inputminted{cpp}{/Users/Corn/Desktop/Code/ACM_SCL/extra/pb_ds_priority.cpp}
\inputminted{cpp}{/Users/Corn/Desktop/Code/ACM_SCL/extra/ext_rope.cpp}
\inputminted{cpp}{/Users/Corn/Desktop/Code/ACM_SCL/extra/MoreStack.cpp}



\end{spacing}

\end{multicols}
\end{document}
%THE SCL ENDS
